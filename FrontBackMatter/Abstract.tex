\thispagestyle{empty}
\begin{center}
	{\Large\bfseries Abstract}
\end{center}

Additive Manufacturing (AM, also known as 3D printing) is defined by a range of technologies that are capable of translating virtual solid model data into physical models in a quick and easy process.
The high potential of this technology resides in the almost infinite freedom of possible geometries of objects being produced, together with a drastic reduction of manufacturing costs and times.

Developed in the 1980s, additive manufacturing now stands as a term grouping several independent processes, whose applications range over a number of different fields.  
A strength of AM techniques is the variety of materials that can be processed, from metals and related compounds to thermoplastics and ceramics.

Despite the widespread interest for AM, there is still much room for improvement, since several fundamental phenomena and aspects are not yet fully understood.
To this end, computational modeling plays a relatively important role and different approaches can shed light on the intrinsic complexity of AM.

The first goal of this research project will be making use of atomistic simulations to study the out--of--equilibrium phenomena that materials undergo during AM processing. After assessing the good reliability of simple semi--empiric potentials, state of the art computational methods will be used to investigate fundamental bulk properties of alloys mostly used in industry. These properties are key parameters that enter meso--scale models employed to bridge theoretical approaches with industrial applications.


Second target of the project will be the study of properties related to coexistence of multiple phases, since during an AM process the material always undergoes a phase transition. In order to investigate more complex properties, such as the anisotropy of solid--liquid interface free energy, new reliable and accurate methods are to be developed.
A promising approach is represented by Neural Networks, which rely on an highly transferable machine learning method capable of providing advanced interatomic potentials, specific for the systems being studied. Results obtained with this method show an accuracy comparable to \textit{ab initio} methods at the cost of force fields potentials.




