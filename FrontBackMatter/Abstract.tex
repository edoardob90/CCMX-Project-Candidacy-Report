\thispagestyle{empty}
\begin{center}
	{\Large\bfseries Abstract}
\end{center}

Additive Manufacturing (AM, also known as 3D printing) is defined by a range of technologies that are capable of translating virtual solid model data into physical models in a quick and easy process. In additive manufacturing the model is broken down into a series of 2D layers of finite thickness, which are then fed into AM machines and combined in a layer--by--layer sequence to form the physical part. The high potential of this technology resides in the almost infinite freedom in possible geometries of objects being produces, together with a drastic reduction of manufacturing costs and time.

Early AM processes and materials were developed in the 1980s and during the decade of 2000s the term ``additive manufacturing'' gained wider currency as an umbrella term grouping independent processes such as \textit{selective laser melting}, \textit{laser sintering} and \textit{laser metal deposition}.