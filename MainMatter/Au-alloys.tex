% !TEX root = ../Main.tex
%!!!!!!!!!!!!!!!!!!!!!%
%%%%% OLD VERSION %%%%%
%!!!!!!!!!!!!!!!!!!!!!%
%
%%%\subsubsection{Gold and its ``colours''}
%%%From ancient times, gold has been used mostly in decorative items, and the colour of gold plays an important role in this application field. Gold and copper are the only two metals that exhibit colour and, thanks to this property, gold alloys can be made to assume a range of colours by varying the alloying additions. Human history is rich in examples of the usage of gold and its colours for different purposes: silver--gold (roughly 70\% of gold and 30\% of silver) was employed by Lydians to make coins and red gold was used by goldsmiths in pre--Columbian cultures\footnote{Ref.~3 Cretu1999}. Today, various ``colours'' of gold are available with the main objective of making jewellery.
%%%
%%%The formation of colour in metallic elements and their alloys can be explained by means of the band theory. When metal and light interact, electrons from the metal surface situated either below or on the Fermi level absorb photons and are promoted to excited states. The efficiency of the absorption and re--emission of light depends on the atomic orbitals from which the energy band originated. In the case of gold and copper, the efficiency decreases with increasing energy, resulting in yellow reflectivity (i.e.\ a larger contribution to reflection arises from the low--energy wavelengths of the light spectrum). In silver, in spite of a very similar electronic structure, transition of electrons above the Fermi level requires energy in excess of that of the violet end of the visible spectrum: all the visible spectrum is evenly reflected, resulting in the characteristic gray metallic color\footnote{Ref.~Cretu1999}.
%%%
%%%Alloying additions to gold and copper can create various colours. It is less well known, however, that gold alloys can have colours which are surprisingly different to the conventional alloys, as in the case of blue, black or purple gold alloys.
%%%Gold coloured alloys can be classified in three main categories:
%%%\begin{enumerate}
%%%    \item Au--Ag--Cu system;
%%%    \item Intermetallic compounds;
%%%    \item Surface oxide layers\footnote{This category will not be discussed here, but details can be found in referenced literatures. For example, see Refs.~47, 48, 49 and 50 of Cretu1999.}.
%%%\end{enumerate}
%%%
%%%%We will discuss in more detail the first and describe briefly intermetallic compounds, while leaving the third class to the referenced literature\footnote{See, for example, Refs.~47, 48, 49 and 50 of Cretu1999.}.
%%%
%%%\paragraph{Au--Ag--Cu system}
%%%This system is the basis of the most used gold alloys with applications in jewellery and, though less and less common, dental prosthesis. Colours of yellow, green and red can be obtained with different ratios of Au:Ag:Cu (see Fig.~\ref{fig:AuAgCu}).
%%%\begin{figure}[tb]
%%%    \centering
%%%    \includegraphics[width=.65\textwidth]{Ag-Au-Cu-colours}
%%%    \caption{Approximate colours of Au--Ag--Cu alloys, which are commonly used in jewelry making. From \href{https://commons.wikimedia.org/wiki/File:Ag-Au-Cu-colours.png}{\ttfamily Wikimedia}.}
%%%    \label{fig:AuAgCu}
%%%\end{figure}
%%%White gold is also based on this system, by adding alloying elements known for their bleaching characteristics such as nickel, palladium or manganese.
%%%
%%%Generally, additions of copper give a reddish tint to the alloy, and additions of silver make the alloy greenish, in accordance with band theory\footnote{The addition of silver to the Au--Cu alloy causes a widening of the energy gap that the electrons have to overcome to reach an energy state above the Fermi level. The wider the gap, the higher the energy absorbed from the incident light, and therefore the reflectivity increases also for the green region of the spectrum.}. An extensive study of the ternary alloy Au--Ag--Cu and its metallurgy, including order--disorder phenomena, can be found in REFS\footnote{Refs.~13,14,15,16 of Cretu1999.}.
%%%
%%%\paragraph{Intermetallic compounds}
%%%The term ``intermetallic compounds'' indicates a particular group of materials whose properties are much different from the individual metallic components. These compounds can be defined as an intermediate phase in an alloy system, with a narrow range of homogeneity and rather simple stoichiometry.
%%%
%%%In this category fall some of the less known coloured golds, such as purple gold and blue gold. Among these, the best known is \ce{AuAl2}, which is formed by about 80\% of gold and 20\% of aluminum\footnote{Ref.~34 Cretu1999} and exhibits a light purple colour, while \ce{AuIn2} and \ce{AuGa2} are used for their clear blue hues.
%%%
%%%The intermetallic compounds behave in some ways
%%%like pure metals, which makes it possible to calculate their band structures, but they are usually brittle materials and their use in traditional jewellery virtually impossible, though they can be faceted and used as gemstones or inlays.
%%%
%%%
%%%
%%%%\paragraph{Surface oxides}


\subsection{Gold alloys}
Gold is an element with one of the most unique spectrum of properties and for this reason its use is particularly widespread, not only for luxury goods like jewelry, but also in many other technical applications despite its high price.
From the chemical point of view it is the most “noble” of all metals: it does not corrode or oxidize and its stability towards influences from a variety of external conditions (be it natural, biomedical or technical) persists up to high temperatures. Moreover, it possesses high electrical and thermal conductivity and, from a mechanical point of view, is very soft, highly malleable and ductile at room temperature. Also optical properties play a crucial role for its applications: in its pure form gold shows a rich yellow color, but gold alloys can be made to assume a range of colours by varying the alloying additions. Human history is rich in examples of the usage of gold and its colours for different purposes.

In almost all areas of application, however, gold cannot be used in its pure form. It needs to be alloyed with other elements, usually metals, to adjust specific properties. For most applications, pure gold is simply too soft. For several applications, the properties of the pure gold need to be maintained as much as possible, for example, bonding wires in electronics or high carat gold jewelry\footnote{Caratage is defined as the fraction of pure gold present in a certain alloy. That is: 24 carat gold (24kt) = 100 wt\% Au, while $X$kt = $(X\times 100)/24$ wt\% Au.}.

The starting point for any kind of metallurgical processing not only for gold is provided by phase diagrams. The most common and used binary and ternary alloys of gold have been extensively studied and reviewed~\cite{OkamotoBOOK1987,PrinceBOOK1990,TernaryAlloyBOOK2006}. In this section we try to give a summary of the metallurgical approaches of gold alloying adopted in many fields of applications, with particular focus to grain refinement and solution strengthening.


\subsubsection{Grain refinement}
Grain refinement or grain--size control is the fundamental process applied routinely to all industrial applications of gold alloys. The list of advantages of a fine--grained material over a coarse--grained one is long and well documented:
\begin{itemize}
    \item increase of strength, malleability and ductility, as well as work--hardening effectiveness during cold--working process;
    \item increase of chemical homogeneity due to less pronounced alloying elements segregation. This in turn has a positive impact on corrosion resistance and susceptibility to embrittlement and crack formation in casting processes;
    \item an improvement of surface characteristics and finishing properties, mainly of interest for decorative and jewelry applications.
\end{itemize}

The only disadvantage provided by fine--grained material occurs at high temperatures where a loss of strength and reduction of creep resistance, due to a change of prevailing deformation mechanisms.

Grain--size control during material processing involves two particularly important steps: \textbf{solidification} after melting and \textbf{annealing} after cold--working.

Grain refinement during solidification mainly is concerned with promoting the formation of a larger number of nuclei for crystal growth from the molten state. This can be achieved by additions of particular elements with higher melting points with respect to the metal and low solubility in solid gold, such as iridium, ruthenium and rhenium, to cite a few~\cite{Nielsen1966}.
The required amount of the additions depends on alloy composition and can vary between \num{50} and \SI{1000}{ppm}. Higher amounts are not meaningful, since no further grain refining effect is usually observed.

Grain refining additions usually are introduced via carefully prepared base metal master alloys, for example Cu--10\% Ir, which need to offer a good liquid and solid solubility for the grain refiner, to provide a homogeneous distribution in the master alloy.

Apart from alloy additions, grain size during solidification is also influenced by processing~\cite{Ott1981}. Cold pouring of molten metal is frequently applied, keeping both metal and mould at low temperatures. This enhances crystal nucleation and retards crystal growth, leading to more fine--grained castings. Mould agitation or vibration support the homogeneous distribution of nuclei and ``mechanically'' refine the structure by breaking up the branches (dendrites) of growing crystals. 


During cold--working processes, the microstructure of a metal becomes strongly distorted: the material is strengthened and ductility is reduced to very low levels after heavy deformation. Complete or partial annealing then is carried out to recover ductility. For sufficiently high annealing temperature and time, the material recrystallizes, which involves nucleation and growth of new grains of regular shape.

Grain refinement during the annealing stage after deformation is mainly based on two mechanisms: increase of the number of recrystallization nuclei in the deformed microstructure and, more importantly, the decrease of grain growth velocity. For gold alloys, some of the additions that have been mentioned above as grain refiners during solidification can also promote the formation of recrystallization nuclei. However, if the growth of the nuclei is not inhibited as well, the described mechanism alone does not necessarily lead to a fine--grained microstructure. In extreme cases it may even have an adverse effect on grain size, namely if excessive growth of a few number of preferentially formed nuclei occurs. Furthermore, a fine--grained recrystallized structure energetically is not stable and tends to coarsen during prolonged annealing.

For all these reasons, the decrease of grain growth velocity by specific alloying additions is always of high importance. Impurities as well as alloying additions with low solubility in the metal matrix can reduce the grain boundary mobility drastically~\cite{Humphreys2004,Lucke1957}, because they tend to segregate to grain boundaries and therefore need to be able to move together with them during recrystallization and grain growth. This requires time and temperature--dependent diffusion processes to occur, which slows down recrystallization and grain growth kinetics.




\subsubsection{Strengthening mechanisms}
Deformation of any metal takes place by the sliding of crystal planes over each other through the movement of dislocations. Any distortion of the crystal lattice or any obstacle in the lattice increases the force required to move the dislocations through the lattice. Hence this leads to an increase of hardness or strength but usually also to a decrease of deformability. \Cref{fig:Au-hardness} gives an overview of the hardness increase in binary gold alloys depending on the type and amount of alloying element added.
\begin{figure}[bt]
\begin{center}
\includegraphics[height=.45\textheight]{Au-hardness}
\caption{Hardness trends in binary gold alloys depending on the type and amount of addition in weight \%~\cite{MetalPocketbook1995}. Units on hardness are expressed according to the ``Vickers hardness test'', that is: $HV=\text{force}/\text{area} = [\si{kgf}]/[\si{\square\milli\metre}]$.}
\label{fig:Au-hardness}
\end{center}
\end{figure}

The largely different dependencies are related to the different prevailing strengthening mechanisms, as well as the particular microstructural condition the material is in. Among these mechanisms, we mention: \textit{solid solution hardening}, \textit{order--disorder transformation hardening} and \textit{precipitation hardening}.

The last two mechanisms are reversible processes, since the material can be brought back to the ``nonhardened'' state by a homogenization heat treatment. These mechanisms are also usually referred to as \textit{age hardening}.

For the sake of completeness, we also cite the irreversible mechanism of \textit{dispersion hardening}, which is based on the irreversible formation of finely dispersed particles, mainly oxides (but also carbides or borides), by reaction of alloying elements with oxygen (or carbon/boron). This often involves special processing routes like internal oxidation or powder metallurgy.

\paragraph{Solid solution hardening.} This mechanism involves dissolution of alloying elements in solid gold, which can either replace gold atoms in the crystal lattice (as substitutional defects) or can fill in the small gaps between gold atoms in the crystal lattice (interstitial solid solution hardening).

The effect of solid solution hardening increases with the extent of lattice distortion associated with an alloying addition; thus the larger is the difference in atomic size between the host metal and the added element, the more prominent is this effect. This partially explains qualitatively the dependence of hardness on the Ag-Cu ratio in the Au-Ag-Cu ternary system, which forms the basis for many common jewelry (mainly for gold ``color spectrum'', \cref{fig:AuAgCu}) and dental alloys.
\begin{figure}[tb]
    \centering
    \includegraphics[width=.6\textwidth]{Ag-Au-Cu-colours}
    \caption{Approximate colours of Au--Ag--Cu alloys, which are commonly used in jewelry making. From \href{https://commons.wikimedia.org/wiki/File:Ag-Au-Cu-colours.png}{\ttfamily Wikimedia}.}
    \label{fig:AuAgCu}
\end{figure}

Gold is only slightly larger than silver, while copper atoms are $\sim 12\%$ smaller than gold: this explains why Cu is more effective in in strengthening of gold by subsitutional solid solution. On the other hand, Ag is completely soluble in Au at all temperatures and for any composition, while Cu is only miscible down to $\sim\SI{410}{\celsius}$: below this point, some intermetallic phases form, which give rise to additional hardening effects by precipitation hardening. Besides Ag and Cu, also palladium and platinum (soluble at any composition only at high temperatures) are largely used to induce solid solution strengthening. However, similarly to Ag, the strengthening effect of Pd and Pt is only small or moderate and usually further treatments are necessary to obtain the alloy with desired properties.
    




\paragraph{Order--disorder transformation hardening.} Due to the lowering of temperature during alloy processing, the atoms can arrange themselves in an ordered solid solution (also referred to as intermetallic compounds or intermediate phases), where different atoms occupy strictly defined lattice sites. Here strengthening is caused by the related elastic distortions, but also because the movement of dislocations is considerably more difficult because the ordered state needs to be preserved.

In the Au-Cu system, the intermediate phases responsible of this mechanism are \ce{CuAu}, \ce{CuAu3} and \ce{Cu3Au}.
The most important one is the phase occurring below \SI{410}{\celsius} at the composition for which the ratio Au:Cu is 1: alternating layers of Au and Cu form a face--centered tetragonal phase (\ce{AgCu} I) stable below \SI{385}{\celsius}, while an orthorombic phase is stable between \SI{385}{\celsius} and \SI{410}{\celsius} (\ce{AgCu} II). When Au:Cu ratio approaches 1:3, two versions of \ce{AuCu3} form with another ordered face--centered cubic symmetry (Cu atoms on lattice faces and Au atoms on lattice edges).

This hardening mechanism is a very important feature for many carat gold jewelry~\cite{Chaston1971:31GoldBook} and dental alloys~\cite{Laberge1979:35GoldBook}. It can be obtained either by slow cooling from high temperatures or by an aging period in the temperature range between \numlist{150;400} \si{\celsius}.
\begin{figure}[t]
    \centering
    \includegraphics[width=.7\textwidth]{Au-order-disorder}
    \caption{(a) Crystal structure of the disordered solid solution in FCC \ce{AuCu}. (b) Crystal structure of the ordered face--centered tetragonal phase, \ce{AgCu}~\cite{Suss2004}.}
    \label{fig:order-disorder}
\end{figure}






\paragraph{Precipitation hardening.} 
The prerequisite for the activation of this mechanism is the reduced solid solubility of an alloying element in the metal matrix. Solid solubility generally decreases with temperature, thus this class of alloying elements can be precipitated by particular heat treatment as finely dispersed particles.

For binary gold alloys, this is the case of Au with Co, Cr, Fe, Mn and many others~\cite{OkamotoBOOK1987}. It can occur for several reasons:
\begin{itemize}
    \item the system behaves as a series of solid solutions at high temperatures, but as temperature decreases a miscibility gap opens up for certain compositions. This is the case of the Au-Ni, Au-Pt systems;
    \item systems in which a limited mutual solubility is present over the whole temperature range, like Au-Co and Au-Cr;
    \item alloy systems that present similar features as above, but where more than one intermetallic phase can occur.
\end{itemize}

Of course, also in more complex systems, like ternary or quaternary alloys, this hardening mechanism can occur. In all the cases, the system is subject to both an heat treatment and an aging period, which should comprise the following steps:
\begin{enumerate}
    \item an homogenization phase useful to dissolve all the alloying elements in the gold matrix. For the example systems mentioned above, this means a temperature between \numlist{700;900} \si{\celsius} for several hours depending on the actual composition;
    \item rapid quenching from the temperature of the previous step, to suppress the formation of precipitates. Up to this phase, the alloying elements remain dissolved in the crystal lattice (in a sort of ``supersaturated'' solid solution);
    \item an aging period at low temperatures (\num{200}--\num{400} \si{\celsius}) that can span from minutes to several hours, depending on the composition and other factors.
\end{enumerate}

Analysis of precipitation hardening phenomena is possible through high resolution electron microscopy, which often reveals that shape, composition and crystal structure of precipitates change with the aging time; thus the last step of the aforementioned schematic recipe for strengthening by precipitation is usually a complex multistage effect. Lastly, prolonged aging almost always leads to to coarsening and coalescence of the precipitates and a related hardness drop, a phenomenon that is referred to as overaging by Ostwald ripening.









 